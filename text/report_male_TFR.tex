% Options for packages loaded elsewhere
\PassOptionsToPackage{unicode}{hyperref}
\PassOptionsToPackage{hyphens}{url}
%
\documentclass[
]{article}
\usepackage{amsmath,amssymb}
\usepackage{iftex}
\ifPDFTeX
  \usepackage[T1]{fontenc}
  \usepackage[utf8]{inputenc}
  \usepackage{textcomp} % provide euro and other symbols
\else % if luatex or xetex
  \usepackage{unicode-math} % this also loads fontspec
  \defaultfontfeatures{Scale=MatchLowercase}
  \defaultfontfeatures[\rmfamily]{Ligatures=TeX,Scale=1}
\fi
\usepackage{lmodern}
\ifPDFTeX\else
  % xetex/luatex font selection
\fi
% Use upquote if available, for straight quotes in verbatim environments
\IfFileExists{upquote.sty}{\usepackage{upquote}}{}
\IfFileExists{microtype.sty}{% use microtype if available
  \usepackage[]{microtype}
  \UseMicrotypeSet[protrusion]{basicmath} % disable protrusion for tt fonts
}{}
\makeatletter
\@ifundefined{KOMAClassName}{% if non-KOMA class
  \IfFileExists{parskip.sty}{%
    \usepackage{parskip}
  }{% else
    \setlength{\parindent}{0pt}
    \setlength{\parskip}{6pt plus 2pt minus 1pt}}
}{% if KOMA class
  \KOMAoptions{parskip=half}}
\makeatother
\usepackage{xcolor}
\usepackage[margin=1in]{geometry}
\usepackage{graphicx}
\makeatletter
\newsavebox\pandoc@box
\newcommand*\pandocbounded[1]{% scales image to fit in text height/width
  \sbox\pandoc@box{#1}%
  \Gscale@div\@tempa{\textheight}{\dimexpr\ht\pandoc@box+\dp\pandoc@box\relax}%
  \Gscale@div\@tempb{\linewidth}{\wd\pandoc@box}%
  \ifdim\@tempb\p@<\@tempa\p@\let\@tempa\@tempb\fi% select the smaller of both
  \ifdim\@tempa\p@<\p@\scalebox{\@tempa}{\usebox\pandoc@box}%
  \else\usebox{\pandoc@box}%
  \fi%
}
% Set default figure placement to htbp
\def\fps@figure{htbp}
\makeatother
\setlength{\emergencystretch}{3em} % prevent overfull lines
\providecommand{\tightlist}{%
  \setlength{\itemsep}{0pt}\setlength{\parskip}{0pt}}
\setcounter{secnumdepth}{-\maxdimen} % remove section numbering
\usepackage{bookmark}
\IfFileExists{xurl.sty}{\usepackage{xurl}}{} % add URL line breaks if available
\urlstyle{same}
\hypersetup{
  pdftitle={Report: male TFR},
  pdfauthor={Henrik-Alexander Schubert},
  hidelinks,
  pdfcreator={LaTeX via pandoc}}

\title{Report: male TFR}
\author{Henrik-Alexander Schubert}
\date{2025-10-03}

\begin{document}
\maketitle

\section{Male Fertility
approximation}\label{male-fertility-approximation}

In the paper by Keilman et al.~(2014), the male fertility rates is
approximated throught a regression model that uses the TFR for women and
the sex ratio in the population aged 20 to 39 as predictors.

\[
log(TFR_m)=\alpha + \beta_1 log(TFR_w) - \beta_2 log(SR_{20-39}) + \epsilon
\]

The authors obtained the following expression after rearranging, because
the two regression coefficients \(\beta_1\) and \(\beta_2\) were not
statistically significant.

\[
TFR_m=0.971 \cdot \frac{TFR_w}{SR_{20-39}}
\]

\subsection{The re-estimation}\label{the-re-estimation}

I have re-estimated the male TFR approximation using male fertility data
from the Human Fertility Collection (Dudel, 2021), Robert Schoen (1985),
a survey-based estiamte by Schoumaker (2019), and my own subnational
male fertility collection (Schubert, 2025). There are a few differences
to those in Keilman et al.~(2014): 1. The coefficients are statistically
significant: This might be related to the larger sample size
(\texttt{r}) and/or the better fit (R-squared=\texttt{r}). Therefore, we
cannot use the reduced version, but need the complete regression
equation for the approximation. 2. Better fit: As mentioned the before,
the fit of the regression model bettern than in the original
publication. Our \(R^2\) is at 0.967, while the \(R^2\) in the original
publication was only 0.83.

The regression results are the following:

\begin{verbatim}
## 
## Call:
## lm(formula = log(tfr_male) ~ log(tfr_female) + log(asr), data = fert_global)
## 
## Residuals:
##      Min       1Q   Median       3Q      Max 
## -0.18069 -0.04411 -0.00745  0.03433  1.67170 
## 
## Coefficients:
##                  Estimate Std. Error t value Pr(>|t|)    
## (Intercept)     -0.095978   0.002813  -34.12   <2e-16 ***
## log(tfr_female)  1.186982   0.003747  316.77   <2e-16 ***
## log(asr)        -0.887622   0.019814  -44.80   <2e-16 ***
## ---
## Signif. codes:  0 '***' 0.001 '**' 0.01 '*' 0.05 '.' 0.1 ' ' 1
## 
## Residual standard error: 0.06941 on 4392 degrees of freedom
## Multiple R-squared:  0.9666, Adjusted R-squared:  0.9666 
## F-statistic: 6.363e+04 on 2 and 4392 DF,  p-value: < 2.2e-16
\end{verbatim}

\[
log(TFR_m)=-0.096 + 1.187 \cdot log(TFR_w) - 0.88 \cdot SR_{20-29}
\]

The regression is visiualized in the figure below:

\begin{figure}
\centering
\pandocbounded{\includegraphics[keepaspectratio]{../results/male_fertility_approx_tfr_fem.pdf}}
\caption{Relationship between female TFR and male TFR. Dashed line is
the estimated regression line from the male TFR approximation. Solid
line is a simple diagonal reflecting unity between the TFR for women and
men.}
\end{figure}

\section{Standardization}\label{standardization}

Standardization is a classical demographic tool to emphasize the impact
of a difference in population structure on aggregate demographic
measures (see, Preston et al.~2001). In this case, we use the method to
show the impact of the difference in the male population structure to
the female population structure on fertility rates. The estimation looks
as follows

\[
TFR_m = \sum_{x=15}^{49} \frac{B_f(x)}{P_m(x)}
\]

Thus, this standardization takes the distribution of births by age of
mother and applies it to the male population structure. This implicitly
assumes that men and women have the same fertility schedule, which is
not true (see Paget et al.~1991, Schoumaker 2019). Normally, men have a
fertility schedule that is shifted to later ages, have a wider
reproductive period into older ages, and have a more gradual decline of
fertility rates after the age mode of childbearing. *Nevertheless, the
standardization reveals the impact of male skewed sex ratios at
reproductive age. Below is the result measured in relative difference
between male and female TFR estiamted by
\(diff=\frac{TFR_m-TFR_w}{TFR_w}\).

\begin{figure}
\centering
\pandocbounded{\includegraphics[keepaspectratio]{../results/standard_east_asia_rel.pdf}}
\caption{Relative difference in the male TFR to the female TFR over time
using the standardization procedure.}
\end{figure}

\end{document}
